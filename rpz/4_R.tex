\sect{Исследовательская часть}
\label{cha:research}
В данном разделе будет приведен анализ реализаций ...

%=============================================================================================
\subsect{Технические характеристики}
Технические характеристики устройства, на котором осуществлялся анализ реализаций алгоритмов:
\begin{itemize}
	\item операционная система -- Windows 10;
	\item оперативная память -- 16 Гб;
	\item процессор -- AMD Ryzen 7 4700U with Radeon Graphics;
	\item количество физических ядер -- 8;
	\item количество логических ядер -- 8.
\end{itemize}

Так как анализ проводился на ноутбуке, то для корректного замера времени ноутбук был подключен в сеть электропитания. Во время провидения анализа была обеспечена стабильная загруженность системы.

%=============================================================================================
\subsect{Демонстрация работы программы}
На рисунке \ref{./img/demo} продемонстрирована работа программы при ручном вводе строки.
%\fig[0.8]{./img/demo}{Демонстрация работы программы}

%=============================================================================================
\subsect{Анализ времени выполнения реализаций алгоритмов}
Замеры времени работы реализованных алгоритмов для каждого эксперимента проводились 10 раз. 
В таблице \ref{tab:time} представлен результат зависимости времени работы реализованных алгоритмов от оазмера строки. В таблице \ref{tab:par} представлен результат зависимости времени работы реализованных алгоритмов от количества потоков.

\renewcommand{\thetable}{\thesubsection.\arabic{table}}
\begin{table}[H]
	\begin{center}
		\begin{center}
			\caption{\label{tab:timeParallel}Результаты замеров времени реализованных алгоритмов в микросекундах}
		\end{center}
		\begin{tabular}{|c|c|c|c|}
			\hline \specialcell{Количество потоков} & \specialcell{Последовательный\\алгоритм}  \\\hline
		 1 & 4056   \\ \hline
		 2 & 2557   \\ \hline
		 4 & 1836   \\ \hline
		\end{tabular}
	\end{center}
\end{table}

На рисунке \ref{./img/time} представлена зависимость времени работы реализованных алгоритмов от размера строки.
%\fig[0.8]{./img/time}{Зависимость сортировок перемешиванием, плавной и Шелла от длины случайных массивов}

%=============================================================================================
\subsect{Вывод}
Параллельная реализация алгоритма Бойера~--~Мура на 8 потоках быстрее последовательной при поиске подстроки в строке размером 500 КБ в 3 раза. При этом, максимальное увеличение скорости параллельной версии алгоритма было достигнута на количестве потоков, равным количеству логических ядер системы, т.е. 8.
