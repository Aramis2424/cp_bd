\sect{Исследовательская часть}
\label{cha:research}
В данном разделе будет приведен анализ реализаций ...

%=============================================================================================
\subsect{Технические характеристики}
Технические характеристики устройства, на котором осуществлялся анализ реализаций алгоритмов:
\begin{itemize}
	\item операционная система -- Windows 10;
	\item оперативная память -- 16 Гб;
	\item процессор -- AMD Ryzen 7 4700U with Radeon Graphics;
	\item количество физических ядер -- 8;
	\item количество логических ядер -- 8.
\end{itemize}

Так как анализ проводился на ноутбуке, то для корректного замера времени ноутбук был подключен в сеть электропитания. Во время провидения анализа была обеспечена стабильная загруженность системы.

%=============================================================================================
\subsect{Демонстрация работы программы}
%На рисунке \ref{./img/demo} продемонстрирована работа программы при ручном вводе строки.
%\fig[0.8]{./img/demo}{Демонстрация работы программы}

%\renewcommand{\thetable}{\thesubsection.\arabic{table}}
%\begin{table}[H]
%	\begin{center}
%		\begin{center}
%			\caption{\label{tab:timeParallel}Результаты замеров времени реализованных алгоритмов в микросекундах}
%		\end{center}
%		\begin{tabular}{|c|c|c|c|}
%			\hline \specialcell{Количество потоков} & \specialcell{Последовательный\\алгоритм}  \\\hline
%		 1 & 4056   \\ \hline
%		 2 & 2557   \\ \hline
%		 4 & 1836   \\ \hline
%		\end{tabular}
%	\end{center}
%\end{table}
