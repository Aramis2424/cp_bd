\sect{Аналитическая часть}
\label{cha:A}
В данном разделе приведены описания методов решения задачи коммивояжера.

%=======================================================================================================================
\subsect{Метод полного перебора}
Метод полного перебора для решения задачи коммивояжера предполагает рассмотрение всех возможных путей в графе и выбор кратчайшего из них.
Преимущество алгоритма -- гарантия нахождения кратчайшего пути; недостаток -- большая трудоемкость — O(n!)~\cite{ulianov}.

%=======================================================================================================================
\subsect{Метод муравьиного алгоритма}

Муравьиный алгоритм \cite{shtovba} --- метод решения задачи оптимизации, основанный на принципе поведения колонии муравьев.

Муравьи действуют, руководствуясь органами чувств. 
Каждый муравей оставляет на своем пути феромоны, чтобы другие могли ориентироваться. 
При большом количестве муравьев наибольшее количество феромона остается на наиболее посещаемом пути, посещаемость же может быть связана с длинами ребер.

Муравей имеет следующие характеристики:
\begin{enumerate}[label=\arabic*)]
	\item зрение --- способность определить длину ребра;
	\item память --- способность запомнить пройденный маршрут;
	\item обоняние --- способность чуять феромон.
\end{enumerate}


Целевая функцию \eqref{d_func}, характеризующая привлекательность ребра, определяемую благодаря зрению:

\begin{equation}
	\label{d_func}
	\eta_{ij} = 1 / D_{ij},
\end{equation}
где $D_{ij}$ — расстояние от текущего пункта $i$ до заданного пункта $j$.


Формула вычисления вероятности перехода в заданную точку (\eqref{posib}):

\begin{equation}
	\label{posib}
	p_{k,ij} = \begin{cases}
		\frac{\eta_{ij}^{\alpha}\cdot\tau_{ij}^{\beta}}{\sum_{q\notin J_k} \eta^\alpha_{iq}\cdot\tau^\beta_{iq}}, j \notin J_k \\
		0, j \in J_k
	\end{cases}
\end{equation}
где $a$ --- параметр влияния длины пути, $b$ --- параметр влияния феромона, $\tau_{ij}$ --- количество феромонов на ребре $ij$, $\eta_{ij}$ --- привлекательность ребра $ij$, $J_k$ --- список посещенных за текущий день городов.

Формула обновления феромона ночью (\eqref{update_phero_1}):
\begin{equation}
	\label{update_phero_1}
	\tau_{ij}(t+1) = \tau_{ij}(t)\cdot(1-p) + \Delta \tau_{ij}(t).
\end{equation}
При этом
\begin{equation}
	\label{update_phero_2}
	\Delta \tau_{ij}(t) = \sum_{k=1}^N \Delta \tau^k_{ij}(t),
\end{equation}
где
\begin{equation}
	\label{update_phero_3}
	\Delta\tau^k_{ij}(t) = \begin{cases}
		Q/L_{k}, \textrm{ребро посещено муравьем $k$ в текущий день $t$,} \\
		0, \textrm{иначе},
	\end{cases}
\end{equation}
где $L_{k}$ -- длина пути k-ого муравья, Q -- константа порядка длины путей, N — количество муравьев.

Необходимо, чтобы вероятность (\ref{posib}) перехода в заданную точку не была равна нулю. 
Следовательно, необходимо ввести минимальное возможное количество феромона ($\tau_{min}$) и в случае, если $\tau_{ij} (t+1)$ принимает значение, меньшее, то увеличить значение феромона до этой величины.

Путь выбирается по следующей схеме:
\begin{enumerate}
	\item Каждый муравей имеет список запретов --- список уже посещенных городов (вершин графа).
	\item Муравьиное зрение отвечает за эвристическое желание посетить вершину.
	\item Муравьиное обоняние отвечает за ощущение феромона на определенном пути (ребре). При этом количество феромона на пути (ребре) в день $t$ обозначается как $\tau_{i, j} (t)$.
	\item После прохождения определенного ребра муравей откладывает на нем некоторое количество феромона, которое показывает оптимальность сделанного выбора, это количество вычисляется по формуле~\eqref{update_phero_3}.
\end{enumerate}
