%В основе преамбулы лежит: https://github.com/vpunch/gost732
\documentclass[a4paper,14pt]{extarticle}

% кодировка
\usepackage[utf8]{inputenc}
\usepackage[T2A]{fontenc}

% поля
\usepackage[left=30mm,right=15mm,top=20mm,bottom=20mm]{geometry}

% переносы слов
\usepackage[english,russian]{babel}

% шрифт Таймс
\usepackage{tempora}
\usepackage{newtxmath}

% межстрочный интервал
\usepackage[onehalfspacing]{setspace}

% отступ первой строки
\usepackage{indentfirst}
\setlength{\parindent}{1.25cm}

\newcommand{\eline}{%
	\vspace{\baselineskip}
}

% содержание
\usepackage{etoc}
\setcounter{tocdepth}{3}

% Гиперссылки \toc с кликабельностью
\usepackage{hyperref}
\hypersetup{
    colorlinks=true, % false: ссылки в цветных прямоугольниках; true: цветные ссылки
    linkcolor=black,  % цвет внутренних ссылок
    filecolor=black, % цвет ссылок на файлы
    urlcolor=black,   % цвет URL-ссылок
    citecolor=black, % цвет ссылок на библиографию
    linkbordercolor={0 0 0} % цвет прямоугольников вокруг ссылок (черный)
}
\usepackage{titlesec}
%\usepackage{tocloft}
\makeatletter
\renewcommand{\@pnumwidth}{15em} % also try with this commented out
\makeatother

\usepackage[normalem]{ulem}

% вставка pdg страниц (несовместима с некоторыми другими пакетами)
%\usepackage{pdfpages}

%===================================================================================
% глубина нумерации разделов
\setcounter{secnumdepth}{5}

% оформление структурного элемента и приложения
\usepackage{titlesec}
\titleformat{\section}
    [block]                           % форма
    {\filcenter\bfseries\normalsize}  % формат полностью
    {ПРИЛОЖЕНИЕ \thesection}          % метка
    {1em}                             % отступ от метки
    {}                                % код перед телом
% счетчик приложений
\usepackage{totcount}
\newtotcounter{annexcount}
% приложение
\renewcommand{\thesection}{\Asbuk{section}}
\newcommand{\annex}[1]{%
    \stepcounter{annexcount}%
    \setcounter{figure}{0}
    \setcounter{table}{0}
    \setcounter{lstlisting}{0}
    \clearpage
    \section{#1}%
}
 
% скрытый структурный элемент
\newcommand{\hidedstructel}[1]{%
    \clearpage
    \section*{#1}%
}
% структурный элемент
\newcommand{\structel}[1]{%
    \hidedstructel{#1}
    \addcontentsline{toc}{section}{#1}%
}

% оформление раздела
\titleformat{\subsection}[hang]
{\normalfont\fontsize{18}{10}\bfseries\centering}{\thesubsection}{0.5em}{} % расстояние между № и названием
\titlespacing\subsection{\parindent}{\parskip}{1.5ex}  % 21 = 14 * 1.5 т.е. полуторный интервал
\renewcommand{\thesubsection}{\arabic{subsection}}
% раздел
\newcommand{\sect}[1]{%
    \clearpage
    \setcounter{figure}{0}  % сбросить нумерацию внутри раздела
    \setcounter{table}{0}
    \setcounter{lstlisting}{0}
    \subsection{#1}
}

% офромление подраздела
\titleformat{\subsubsection}[hang]
{\bfseries\normalsize}{\thesubsubsection}{0.5em}{}
\titlespacing\subsubsection{\parindent}{1ex}{1ex}
% подраздел
\usepackage{placeins}
\newcommand{\subsect}[1]{%
    \FloatBarrier
    \subsubsection{#1}
    \renewcommand{\theparagraph}{\thesubsubsection.\arabic{paragraph}}
}

% офрмление пункта
\titleformat{\paragraph}[runin]
{\bfseries\normalsize}{\theparagraph}{0.5em}{} 
\titlespacing\paragraph{\parindent}{1ex}{1ex}
% пункт
\newcommand{\parag}{
    \paragraph{}
}

% офрмление подпункта
\titleformat{\subparagraph}[runin]
{\bfseries\normalsize}{\thesubparagraph}{0.5em}{} 
\titlespacing\subparagraph{\parindent}{1ex}{1ex}
% пункт
\newcommand{\subparag}{
    \subparagraph{}
}

%===================================================================================
% перечисления
\usepackage{enumitem}
% общие параметры для списков
\setlist{
    topsep=0pt,                   % отступ сверху и снизу списка
    partopsep=0pt,                % то же самое
    %leftmargin=0pt,               % отступ слева
    labelsep=0pt,                 % отступ метки
    align=left,                 % выравнивание метки
    listparindent=\parindent,   % отступ первой строки абзаца
    itemsep=0pt,                  % отступ между элементами
    parsep=0pt                    % отступ между абзацами и элементами
}
% параметры для нумерованных списков
\setlist[enumerate]{
    label={\arabic*)},
    labelwidth=1.4em%,
    %itemindent=\parindent+\labelwidth
    \setlength{\itemindent}{\labelwidth}
}
% параметры для ненумерованного списка
\setlist[itemize]{
    label=---~,  % в списках тире короткое, в тексте - длинное
    labelwidth=1.2em%,
    %itemindent=\parindent+\labelwidth
    \setlength{\itemindent}{\labelwidth}
}
% параметры для буквенного списка
\AddEnumerateCounter*{\asbuk}{\c@asbuk}{7}
\newlist{asblist}{enumerate}{2}
\setlist[asblist]{
    label={\asbuk*)},
    labelwidth=1.4em%,
    \setlength{\itemindent}{\labelwidth}
}

%===================================================================================
% подписи
\usepackage[singlelinecheck=false]{caption}
\DeclareCaptionLabelSeparator{gost}{~---~}
\captionsetup{labelsep=gost}

% пакет для работы с рисунками
\usepackage{graphicx}
\usepackage{chngcntr}
\graphicspath{{img/}}

% иллюстрация
\DeclareCaptionLabelFormat{gostfigure}{Рисунок #2}
\captionsetup[figure]{justification=centering, labelformat=gostfigure, position=bottom}
\newcommand{\fig}[3][1]{
	\renewcommand{\thefigure}{\thesubsection.\arabic{figure}}
    \begin{figure}[h]
        \centering
        \includegraphics[width=#1\textwidth]{#2}
        \caption{#3}\label{#2}
    \end{figure}
}
% font=singlespacing по умолчанию
%skip=-6pt

\newcommand{\comfig}[3][1]{
	\renewcommand{\thefigure}{\arabic{figure}}
    \begin{figure}[h]
        \centering
        \includegraphics[width=#1\textwidth]{#2}
        \caption{#3}\label{#2}
    \end{figure}
}

\newcommand{\annexfig}[3][1]{
	\counterwithin{figure}{section}
    \begin{figure}[h]
        \centering
        \includegraphics[width=#1\textwidth]{#2}
        \caption{#3}\label{#2}
    \end{figure}
}
%-----------------------------------------------------------------------------------
% листинг
\usepackage{listings}
\usepackage{xcolor}
% общие правила оформления кода
\lstset{ %
	language=SQL,   					% выбор языка для подсветки	
	basicstyle=\small,			% размер и начертание шрифта для подсветки кода
	numbers=left,						% где поставить нумерацию строк (слева\справа)
	numberstyle=\small,					% размер шрифта для номеров строк
	stepnumber=1,						% размер шага между двумя номерами строк
	numbersep=5pt,						% как далеко отстоят номера строк от подсвечиваемого кода
	frame=single,						% рисовать рамку вокруг кода
	tabsize=4,							% размер табуляции по умолчанию равен 4 пробелам
	captionpos=t,						% позиция заголовка вверху [t] или внизу [b]
	breaklines=true,					
	breakatwhitespace=true,				% переносить строки только если есть пробел
	escapeinside={\#*}{*)},				% если нужно добавить комментарии в коде
	xleftmargin=2em, % Увеличение левого отступа
    %framexleftmargin=3.5em, % Увеличение левого отступа для рамки
	backgroundcolor=\color{white}
}

% код в разделе
\DeclareCaptionLabelFormat{custlisting}{Листинг #2}
\captionsetup[lstlisting]{justification=raggedright, labelformat=custlisting, position=top}
\newcommand{\code}[2]{
	\renewcommand{\thelstlisting}{\thesubsection.\arabic{lstlisting}}
    \lstinputlisting[breaklines, caption=#2, label=#1]{#1}
}

% код вне раздела
\newcommand{\comcode}[2]{
	\renewcommand{\thelstlisting}{\arabic{lstlisting}}
    \lstinputlisting[breaklines, caption=#2, label=#1]{#1}
}

% код в приложении
\newcommand{\annexcode}[2]{
	\counterwithin{lstlisting}{section}
	\renewcommand{\thelstlisting}{\thesection.\arabic{lstlisting}}
    \lstinputlisting[breaklines, caption=#2, label=#1]{#1}
}

% Написание кода прямо в документе. Шаблон.
%\begin{lstlisting}[label=some-code,caption={Это крутой исходный код}]
%	\renewcommand{\thelstlisting}{\thesection.\arabic{lstlisting}}
%	<text code>
%\end{lstlisting}
% а ещё можно через \VerbatimEnvironment

%-----------------------------------------------------------------------------------
% таблица
% формат подписи к таблицы указывать самостоятельно перед begin(\tbl) в формате:
% \renewcommand{\thetable}{\arabic{table}}
\newenvironment{tbl}[3]
{
    \begin{table}[h]
        \small
        \centering
        \caption{#2}\label{tbl:#1}
        \begin{tabular}{|#3|}
            \hline
}{
            \hline
        \end{tabular}
    \end{table}
}
\DeclareCaptionLabelFormat{gosttable}{Таблица #2}
\captionsetup[table]{justification=raggedright, labelformat=gosttable, position=top}

\usepackage{caption}
\captionsetup[table]{labelsep=endash}

\newcommand{\specialcell}[2][c]{%
  \begin{tabular}[#1]{@{}c@{}}#2\end{tabular}}

\usepackage{tabularx}

% объединение строк
\usepackage{multirow}
\newcommand{\mr}[2]{\multirow[t]{#1}{=}{#2}}

% колонки
\usepackage{array}
\newcolumntype{M}[1]{>{\centering\arraybackslash}m{#1}}
\newcolumntype{N}[1]{>{\raggedright\arraybackslash}p{#1}}

% заголовок таблицы
\usepackage{xparse}
\NewExpandableDocumentCommand\thead{t< t> O{1} m m}{%
    \IfBooleanTF{#1}{%
        \IfBooleanTF{#2}{%
            \multicolumn{#3}{|M{#4}|}{#5}%
        }{%
            \multicolumn{#3}{|M{#4}}{#5}%
        }
    }{%
        \IfBooleanTF{#2}{%
            \multicolumn{#3}{M{#4}|}{#5}%
        }{%
            \multicolumn{#3}{M{#4}}{#5}%
        }%
    }%
}

% длинная таблица
\usepackage{longtable}
\newenvironment{longtbl}[3]
{
    \small
    \begin{longtable}{|#3|}
        \caption{#2}\label{tbl:#1}\\
        \hline
}{
        \hline
    \end{longtable}
}

% шаблон таблицы
%\begin{table}[hbtp]
%	\begin{center}
%		\begin{flushleft}
%			\caption{\label{tab:compare}Классификация существующих методов определения %заимствований в исходных кодах программ}
%		\end{flushleft}
%		\begin{tabular}{|l | l | l | l | l | l | l | l |} 
%			\hline 
%             ~					& {1}	 &    {2.1} &   {2.2} & {2.3} & {2.4} & {3} & {4} \\ \hline
%			Текст  & Нет  &   100\%  & 80\%    &  76\%   &  65\%    &  \textbf{80\% }& Нет  \\ \hline
%			Токены  &  Да &   100\%  &  92\%   & 87\%  &  74\%  &    \textbf{88\% } & Нет  \\ \hline
%		    Метрики &   Да &   100\%   &  87\%   & 95\%    &  72\%   & \textbf{89\% }& Нет  \\ \hline
%			Деревья   &   Да &    100\%  &  93\%   &  91\%   &  60\%  &  \textbf{84\%}& Нет  \\ \hline
%5			Низкоуровневый код &  Да &    99\%  &  100\%   &   85\%   &   80\%  &  \textbf{91\%}& %Да \\  \hline
%		\end{tabular}
%	\end{center}
%\end{table}

%===================================================================================
% математика и графика

% математика
\usepackage{amsmath}  % amsmath mathtools
\numberwithin{equation}{subsection}

% графики
\usepackage{tikz, pgfplots}
\pgfplotsset{compat=newest}

\usepackage{adjustbox}
\usepackage{float}
\usepackage{url}

%===================================================================================
% источники
\usepackage{csquotes}
\usepackage[%
    backend=biber,
    bibstyle=gost-numeric, % gost-numeric  utf8gost705u.bst ieee
    sorting=none
]{biblatex}
\addbibresource{bibliography.bib}
%\addbibresource{biblatex-examples.bib}
\newcommand{\showbib}{%
    \structel{СПИСОК ИСПОЛЬЗОВАННЫХ ИСТОЧНИКОВ}%
    \printbibliography[heading=none]%
}

% отступы в источниках
\defbibenvironment{bibliography}
    {\list
        {}
        {\setlength{\leftmargin}{0pt}%
         \setlength{\itemindent}{\parindent}%
         \setlength{\itemsep}{0pt}%
         \setlength{\parsep}{0pt}}}
    {\endlist}
    {\item
     \printtext[labelnumberwidth]{%
        \printfield{labelprefix}%
        \printfield{labelnumber}%
     }%
     \hspace{0.5em}}

% метка без точки
\DeclareFieldFormat{labelnumberwidth}{#1.}

%===================================================================================
% прочее

% номер последней страницы
\usepackage{lastpage}

% счетчик источников
\newtotcounter{bibcount}
\AtEveryBibitem{
    \stepcounter{bibcount}%
}
% счетчики таблиц и рисунков
\usepackage{xassoccnt}
\newtotcounter{tblcount}
\DeclareAssociatedCounters{table}{tblcount}
\newtotcounter{figcount}
\DeclareAssociatedCounters{figure}{figcount}

%===================================================================================
%===================================================================================
