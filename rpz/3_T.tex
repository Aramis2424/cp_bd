\sect{Технологический раздел}
\label{cha:T}
В данном разделе приведен выбор средств реализации программного обеспечения, исходные коды создания таблиц и ролей базы данных.

%==============================================================================
\subsect{Выбор системы управления базами данных}
Были выбраны следующие критерии при выборе системы управления базами данных (СУБД):
\begin{enumerate}
	\item СУБД должна быть бесплатной;
	\item СУБД должна поддерживать процедурные расширения языка SQL;
	\item СУБД должна работать на операционных системах Linux и Windows.
\end{enumerate}

В таблице~\ref{tab:subd} приведено сравнение популярных СУБД.
\begin{table}[H]
	\begin{center}
		\begin{center}
			\caption{\label{tab:subd}Сравнение СУБД}
		\end{center}
		\begin{tabular}{|c|c|c|c|}
			\hline 
			СУБД & Критерий \textnumero{} 1 & Критерий \textnumero{} 2 &  Критерий \textnumero{} 3 \\ \hline
			PostgreSQL~\cite{postgres} &  + & + & + \\ \hline
			MySQL~\cite{mysql} &  + &  - & + \\ \hline
			Oracle~\cite{oracle} &   -  &   + & +  \\ \hline
		\end{tabular}
	\end{center}
\end{table}

В результате проведенного анализа в качестве СУБД была выбрана  PostgreSQL.

%==============================================================================
\subsect{Средства реализации}
В качестве средств реализации программного обеспечения были выбраны следующие технологии:
\begin{itemize}
	\item язык программирования -- Java~\cite{java};
	\item СУБД -- PostgreSQL;
	\item расширение языка SQL для написания процедуры -- PL/pgSQL~\cite{plpgsql};
	\item для подключения к базе данных из приложения -- Hibernate~\cite{hibernate}.
\end{itemize}


%==============================================================================
%\subsect{Создание таблиц и установка ограничений целостности}
\subsect{Создание базы данных}
В листингах \ref{./lst/create_tables.sql} и \ref{./lst/constraints.sql} приведены коды создания таблиц, описанных в разделе~\ref{cha:C} и установки ограничений целостности соответственно.

~\\
~

\code{./lst/create_tables.sql}{Создание таблиц}
\code{./lst/constraints.sql}{Установка ограничений целостности}

%==============================================================================
%\subsect{Создание хранимой процедуры}
В листинге \ref{./lst/proc.sql} приведен код для создания хранимой процедуры, спроектированной в разделе~\ref{cha:C}.
\code{./lst/proc.sql}{Создание таблиц}

%==============================================================================
%\subsect{Создание ролей}
В листинге \ref{./lst/role.sql} приведен код для создания ролевой модели, спроектированной в разделе~\ref{cha:C}.
\code{./lst/role.sql}{Создание ролевой модели}

%=============================================================================================
\subsect{Демонстрация работы программы}
На рисунке \ref{./png/demo} продемонстрирован интерфейс главного меню после успешной авторизации.
\fig[0.8]{./png/demo}{Демонстрация работы программы}\\
~\\
~\\
~\\
~\\

%==============================================================================
\textbf{Вывод}

В данном разделе были выбраны средства реализации программного обеспечения, а также приведены исходные коды создания таблиц и хранимой процедуры базы данных.
