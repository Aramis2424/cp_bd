\sect{Технологическая часть}
\label{cha:impl}

В данном разделе будут приведены требования к программному обеспечению (ПО), а также средства реализации и сами реализации алгоритмов умножения матриц.

%==========================================================================
\subsect{Требования к программному обеспечению}
Программа должна предоставлять пользователю возможность ввода произвольной строки любого размера. В качестве результата работы программа должна вывести индексы начала вхождения искомой подстроки в строке или уведомить пользователя о том, что вхождений нет.

%==============================================================================
\subsect{Средства реализации}

В данной работе для реализации был выбран язык программирования Python. 
В данной лабораторной работе требуется замерить процессорное время работы выполняемой программы. 
Инструменты для этого присутствуют в выбранном языке программирования.

Время работы было замерено с помощью функции \textit{process\_time(...)} из библиотеки $time$ \cite{python-lang-time}.


\subsect{Сведения о файлах программы}
Программа состоит из следующих файлов:
\begin{enumerate}
	\item main.py -- файл, содержащий меню;
	\item algorithms.py -- файл, содержащий реализации алгоритма;
	\item test.py -- файл с системой тестирования реализаций алгоритмов.
\end{enumerate}

%==============================================================================
\subsect{Реализация алгоритмов}

В листингах \ref{./lst/full_comb.py} и \ref{./lst/ants} приведены реализации алгоритма полного перебора и муравьиного алгоритма соответственно.

\code{./lst/full_comb.py}{Алгоритм полного перебора}
\code{./lst/ants.py}{Муравьиный алгоритм}

